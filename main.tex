%%%%%%%%%%%%%%%%%
% This is an sample CV template created using altacv.cls
% (v1.3, 10 May 2020) written by LianTze Lim (liantze@gmail.com). Now compiles with pdfLaTeX, XeLaTeX and LuaLaTeX.
% This fork/modified version has been made by Nicolás Omar González Passerino (nicolas.passerino@gmail.com, 15 Oct 2020)
%
%% It may be distributed and/or modified under the
%% conditions of the LaTeX Project Public License, either version 1.3
%% of this license or (at your option) any later version.
%% The latest version of this license is in
%%    http://www.latex-project.org/lppl.txt
%% and version 1.3 or later is part of all distributions of LaTeX
%% version 2003/12/01 or later.
%%%%%%%%%%%%%%%%

%% If you need to pass whatever options to xcolor
\PassOptionsToPackage{dvipsnames}{xcolor}

%% If you are using \orcid or academicons
%% icons, make sure you have the academicons
%% option here, and compile with XeLaTeX
%% or LuaLaTeX.
% \documentclass[10pt,a4paper,academicons]{altacv}

%% Use the "normalphoto" option if you want a normal photo instead of cropped to a circle
% \documentclass[10pt,a4paper,normalphoto]{altacv}

\documentclass[10pt,a4paper,ragged2e,withhyper]{altacv}

%% AltaCV uses the fontawesome5 and academicons fonts
%% and packages.
%% See http://texdoc.net/pkg/fontawesome5 and http://texdoc.net/pkg/academicons for full list of symbols. You MUST compile with XeLaTeX or LuaLaTeX if you want to use academicons.

% Change the page layout if you need to
\geometry{left=1.2cm,right=1.2cm,top=1cm,bottom=1cm,columnsep=0.75cm}

% The paracol package lets you typeset columns of text in parallel
\usepackage{paracol}

% Change the font if you want to, depending on whether
% you're using pdflatex or xelatex/lualatex
\ifxetexorluatex
  % If using xelatex or lualatex:
  \setmainfont{Roboto Slab}
  \setsansfont{Lato}
  \renewcommand{\familydefault}{\sfdefault}
\else
  % If using pdflatex:
  \usepackage[rm]{roboto}
  \usepackage[defaultsans]{lato}
  % \usepackage{sourcesanspro}
  \renewcommand{\familydefault}{\sfdefault}
\fi

% ----- LIGHT MODE -----
\definecolor{SlateGrey}{HTML}{2E2E2E}
\definecolor{LightGrey}{HTML}{666666}
\definecolor{PrimaryColor}{HTML}{001F5A}
\definecolor{SecondaryColor}{HTML}{0039AC}
\definecolor{ThirdColor}{HTML}{F3890B}
\definecolor{BackgroundColor}{HTML}{E2E2E2}
\colorlet{name}{PrimaryColor}
\colorlet{tagline}{PrimaryColor}
\colorlet{heading}{PrimaryColor}
\colorlet{headingrule}{ThirdColor}
\colorlet{subheading}{SecondaryColor}
\colorlet{accent}{SecondaryColor}
\colorlet{emphasis}{SlateGrey}
\colorlet{body}{LightGrey}
\pagecolor{BackgroundColor}   
% ----- DARK MODE -----
%\definecolor{BackgroundColor}{HTML}{242424}
%\definecolor{SlateGrey}{HTML}{6F6F6F}
%\definecolor{LightGrey}{HTML}{ABABAB}
%\definecolor{PrimaryColor}{HTML}{3F7FFF}
%\colorlet{name}{PrimaryColor}
%\colorlet{tagline}{PrimaryColor}
%\colorlet{heading}{PrimaryColor}
%\colorlet{headingrule}{PrimaryColor}
%\colorlet{subheading}{PrimaryColor}
%\colorlet{accent}{PrimaryColor}
%\colorlet{emphasis}{LightGrey}
%\colorlet{body}{LightGrey}
%\pagecolor{BackgroundColor}

% Change some fonts, if necessary
\renewcommand{\namefont}{\Huge\rmfamily\bfseries}
\renewcommand{\personalinfofont}{\small\bfseries}
\renewcommand{\cvsectionfont}{\LARGE\rmfamily\bfseries}
\renewcommand{\cvsubsectionfont}{\large\bfseries}

% Change the bullets for itemize and rating marker
% for \cvskill if you want to
\renewcommand{\itemmarker}{{\small\textbullet}}
\renewcommand{\ratingmarker}{\faCircle}

%% sample.bib contains your publications
\addbibresource{publications.bib}

\begin{document}
    \name{Nicolas Gartner}
    \tagline{Docteur-Ingénieur \newline Mécanique (fluide-solide) Robotique}
    %% You can add multiple photos on the left or right
    \photoL{4cm}{Nicolas_v2}
    
    \personalinfo{
        \email{nicolas.gartner@gmail.com}\smallskip
        \phone{+33 6 28 30 56 96}
        \location{Saint Chamas, France}\\
        \linkedin{ngartner}
        \github{ngartner}
        %\homepage{nicolasomar.me}
        %\medium{nicolasomar}
        %% You MUST add the academicons option to \documentclass, then compile with LuaLaTeX or XeLaTeX, if you want to use \orcid or other academicons commands.
        %\orcid{0000-0003-4068-2889}
        %% You can add your own arbtrary detail with
        %% \printinfo{symbol}{detail}[optional hyperlink prefix]
        % \printinfo{\faPaw}{Hey ho!}[https://example.com/]
        %% Or you can declare your own field with
        %% \NewInfoFiled{fieldname}{symbol}[optional hyperlink prefix] and use it:
        % \NewInfoField{gitlab}{\faGitlab}[https://gitlab.com/]
        % \gitlab{your_id}
    }
    
    \makecvheader
    %% Depending on your tastes, you may want to make fonts of itemize environments slightly smaller
    % \AtBeginEnvironment{itemize}{\small}
    
    %% Set the left/right column width ratio to 6:4.
    \columnratio{0.25}

    % Start a 2-column paracol. Both the left and right columns will automatically
    % break across pages if things get too long.
    \begin{paracol}{2}
        % ----- STRENGTHS -----
        \cvsection{Compétences}
            \cvtag{Programmation}
            
            \cvtag{Calcul scientifique}
            
            \cvtag{Recherche scientifique}
            
            \cvtag{Conception 3D}
            
            \cvtag{Science des données}
            
            \cvtag{Dynamique des fluides}
            
            \cvtag{Ingénierie mécanique}
        % ----- STRENGTHS -----
        
        % ----- LEARNING -----
        \cvsection{Informatique}
            \cvtag{Windows}
            \cvtag{Ubuntu}
            \cvtag{ROS}
            \smallskip
            
            \cvtag{Latex}
            \cvtag{Microsoft Office}
            \smallskip
            
            \cvtag{C++}
            \cvtag{C}
            \cvtag{Python}
            \cvtag{Qt}
            \cvtag{Anaconda}
            \cvtag{OpenMP}
            \cvtag{git}
            \smallskip
            
            \cvtag{Matlab}
            \cvtag{Scilab}
            \cvtag{Pandas}
            \cvtag{matplotlib}
            \smallskip
            
            \cvtag{Gazebo}
            \cvtag{Vortex}
            \cvtag{ADAMS}
            \smallskip
            
            \cvtag{CATIA v5}
            \cvtag{Inventor}
            \cvtag{Solidworks}
            \cvtag{FreeCAD}
            \smallskip
            
            \cvtag{ANSYS Fluent}
        % ----- LEARNING -----
        
        % ----- LANGUAGES -----
        \cvsection{Languages}
            \cvlang{Français}{Langue maternelle} \\
            \smallskip
            \cvlang{Anglais}{Courant \newline TOEIC - 965 (2016)} \\
            \smallskip
            \cvlang{Allemand}{Courant \newline Goethe B2 - 91 (2016)} \\
            \smallskip
            \cvlang{Portugais}{Avancé}
            %% Yeah I didn't spend too much time making all the
            %% spacing consistent... sorry. Use \smallskip, \medskip,
            %% \bigskip, \vpsace etc to make ajustments.
            \smallskip
        % ----- LANGUAGES -----
            
        \newpage
        % ----- REFERENCES -----
        \cvsection{References}
            \cvref{Vincent Hugel}{vincent-hugel-0955b56a}
            \smallskip
            
            \cvref{Niels Montanari}{niels-montanari}
            \smallskip
            
            \cvref{Mathieu Richier}{mathieu-richier-5126a863}
            \smallskip
            
            \cvref{Elisabeth Murisasco}{elisabeth-murisasco-20834210}
            \smallskip
        % ----- REFERENCES -----
        
        % ----- MOST PROUD -----
        % \cvsection{Most Proud of}
        
        % \cvachievement{\faTrophy}{Fantastic Achievement}{and some details about it}\\
        % \divider
        % \cvachievement{\faHeartbeat}{Another achievement}{more details about it of course}\\
        % \divider
        % \cvachievement{\faHeartbeat}{Another achievement}{more details about it of course}
        % ----- MOST PROUD -----
        
        % \cvsection{A Day of My Life}
        
        % Adapted from @Jake's answer from http://tex.stackexchange.com/a/82729/226
        % \wheelchart{outer radius}{inner radius}{
        % comma-separated list of value/text width/color/detail}
        % \wheelchart{1.5cm}{0.5cm}{%
        %   6/8em/accent!30/{Sleep,\\beautiful sleep},
        %   3/8em/accent!40/Hopeful novelist by night,
        %   8/8em/accent!60/Daytime job,
        %   2/10em/accent/Sports and relaxation,
        %   5/6em/accent!20/Spending time with family
        % }
        
        % use ONLY \newpage if you want to force a page break for
        % ONLY the current column
        
        
        %% Switch to the right column. This will now automatically move to the second
        %% page if the content is too long.
        \switchcolumn
        
        % ----- ABOUT ME -----
        %\cvsection{About Me}
        %    \begin{quote}
       %         Lorem ipsum dolor sit amet, consectetur adipiscing elit, sed do eiusmod tempor incididunt ut labore et dolore magna aliqua.
       %     \end{quote}
        % ----- ABOUT ME -----
        
        % ----- EXPERIENCE -----
        \cvsection{Expérience}
            \cvevent{Ingénieur de recherche}{| Université de Toulon}{Déc 2020 -- Aujourd'hui}{Toulon, France}
            \begin{itemize}
                \item Collaboration avec Centroid LAB pour l'exploitation des résultats de thèse en vue d'une mise sur le marché 
                \item Étude de marché
                \item Ajout de nouveaux cas d'essais pour valider les résultats de recherche
            \end{itemize}
            
            \medskip
            
            \cvevent{Enseignant}{| EM Lyon}{Sept 2020 -- Aujourd'hui}{Ecully, France}
            \begin{itemize}
                \item Enseignement du Machine Learning et de Python dans le cadre du master Digital Marketing \& Data Science.
                \item Responsable du cours de Machine Learning pour l'année universitaire 2020-2021 (60 étudiants - 3 groupes)
                \item Création des supports de cours de Machine Learning
                \item Encadrement d'une personne répliquant le contenu du cours
                \item Tutorat de soutient pour le cours de Python
            \end{itemize}
            
            \divider
            
            \cvevent{Consultant scientifique}{| Centroid LAB}{Août 2020 -- Sept 2020}{Los Angeles (distanciel), USA}
            \begin{itemize}
                \item Préparation de simulations de scénario catastrophe dans le cadre du projet open-source de centrale nucléaire Open-100.
                \item Création d'outils support pour le logiciel de simulation Neutrino.
            \end{itemize}
            
            \divider
            
            \cvevent{Doctorant et ingénieur d'études}{| Université de Toulon}{Oct 2016 -- Juin 2020}{Toulon, France}
            \begin{itemize}
                \item Développement d'un partenariat avec l'entreprise CENTROID LAB.
                \item Membre élu au conseil du laboratoire COSMER, de l'école doctorale 548 et du pôle INP de l'université de Toulon
                \item Réalisation d'expérimentations supplémentaire en collaboration avec l'université de Gérone (Espagne) dans le cadre du projet H2020 EU Marine Robots sous un contrat d'ingénieur d'études.
                \item Enseignement de la mécanique vibratoire TD/TP (104h) et de l'automatique TP (24h)
            \end{itemize}
            
            \divider
            
            \cvevent{Stage Ingénieur R \& D}{| Altran}{Avril 2016 -- Sept 2016}{Aix-en-Provence, France}
            \begin{itemize}
                \item Conception robotique : Projet Méthode et Analyse du Démantèlement Nucléaire
                \item Conception et optimisation de la structure du bras robotisé d'un robot mobile capable de réaliser des opération de carottage
            \end{itemize}
            
            \divider
            
            \cvevent{Stage Ingénieur R \& D}{| Mayfran International}{Sept 2015 -- Feb 2016}{Landgraaf, Pays-Bas}
            \begin{itemize}
                \item Développement produit / convoyeur et système de filtration
                \item Création d'outils support pour le logiciel de simulation Neutrino.
            \end{itemize}
            
            \divider
            
            \cvevent{Stage de recherche}{| Universidade Federal Uberlândia}{Févr 2015 -- Août 2015}{Uberlândia, Brésil}
            \begin{itemize}
                \item Fabrication d'échangeur de chaleur par méthode de fabrication additive
                \item Traitement des données d'essais de soudage de métaux
            \end{itemize}
            
            \divider
            
            
        % ----- EXPERIENCE -----
        
        % ----- EDUCATION -----
        \cvsection{Formation}
            \cvevent{Doctorat }{| Université de Toulon}{Oct 2016 -- Juin 2020}{Toulon, France}
            \begin{itemize}
                \item Identification de paramètres hydrodynamiques par simulation avec Smoothed Particle Hydrodynamics.
                \item Simulation de la dynamique des véhicules/robots marins et étude des paramètres hydrodynamiques
                \item Simulation numérique de l'écoulement de fluide avec la technique SPH (modèle incompressible et faiblement compressible)
                \item Utilisation du calcul parallèle pour s'approcher de calcul en temps réel
            \end{itemize}
            \divider
            
            \cvevent{Diplôme d'ingénieur }{| Sigma Clermont - ex IFMA}{Sep 2012 -- Sep 2016}{Clermont-Ferrand, France}
            \begin{itemize}
                \item Spécialisation Machines Mécanismes et systèmes
            \end{itemize}
            
        \cvsection{Publications}
            \nocite{*}
            \printbibliography[heading=none]
            
    \end{paracol}
\end{document}
