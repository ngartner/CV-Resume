\documentclass[11pt,a4paper]{moderncv}
\usepackage[french]{babel}\addto{\captionsfrench}{\renewcommand{\refname}{Bibliographie}}
\usepackage[utf8]{inputenc}
\usepackage{amsmath}
\usepackage{amsfonts}
\usepackage{amssymb}
%Marges et taille des colonnes de dates
\usepackage[left=2cm,right=2cm,top=1.1cm,bottom=1.1cm]{geometry}

\author{Nicolas Gartner}
\title{Curriculum Vitae}

% ma préférence en terme de design
\moderncvtheme[red]{classic}

% Une entête classique
\firstname{Nicolas}
\familyname{Gartner}
\title{Bientôt docteur en robotique disponible en juin 2020}
\extrainfo{27 ans \, Permis B}
\address{9 avenue Seyssaud}{13250 Saint Chamas}
\email{nicolas.gartner@gmail.com}
\mobile{+33 6 28 30 56 96}
\social[linkedin]{www.linkedin.com/in/ngartner}                        % optional, remove / comment the line if not wanted
\social[github]{ngartner}
\photo[70pt][0.4pt]{Nicolas_Gartner.jpg}

\begin{document}
\maketitle
% Marge négative entre le titre et la partie expérience, pour gagner de la place
\vspace*{-3\baselineskip}
\section{Expériences professionnelles}
\cventry{2019-mai 2020}{Ingénieur d'étude}{Laboratoire COSMER}{Université de Toulon}{}{Réalisation d'expérimentations dans le cadre du projet H2020 EU Marine Robots}
\cventry{2017-2019}{Chargé d'enseignement}{Seatech et IUT}{Université de Toulon}{}{Enseignement de la mécanique vibratoire TD/TP (104h) et de l'automatique TP (24h)}
\cventry{2016}{Ingénieur R \& D}{Altran}{Aix-en-Provence}{}{Conception robotique : Projet Méthode et Analyse du Démantèlement Nucléaire (6 mois)}
\cventry{2015-2016}{Ingénieur R \& D}{Mayfran International}{Landgraaf}{Pays-Bas}{Développement produit / convoyeur et système de filtration (6 mois)}
\cventry{2015}{Stage de recherche}{Universidade Federal Uberlândia}{}{Brésil}{Fabrication d’échangeur de chaleur par méthode de fabrication additive (6 mois)}
\section{Formation}
\cventry{2016-2020}{Doctorat}{Université de Toulon}{Laboratoire COSMER}{}{Identification de paramètres hydrodynamiques par simulation avec Smoothed Particle Hydrodynamics. Membre élu au conseil du laboratoire COSMER, de l'école doctorale 548 et du pôle INP de l'université de Toulon}
\cventry{2012-2016}{Diplôme d'ingénieur}{Sigma Clermont - ex IFMA}{Clermont-Ferrand}{}{Spécialisation Machines Mécanismes et systèmes}
%
\section{Langues}
\cvitemwithcomment{Français}{Langue maternelle}{}
\cvitemwithcomment{Anglais}{Courant}{TOEIC - 965 (2016)}
\cvitemwithcomment{Allemand}{Courant}{Goethe B2 - 91 (2016)}
\cvitemwithcomment{Portugais}{Avancé}{}
%
\section{Informatiques}
\renewcommand{\arraystretch}{1}
\setlength{\tabcolsep}{0.25cm}
\begin{tabular}{|l|l|l|l|}
\hline
OS            & Windows, Ubuntu, ROS        & Écriture         & LateX, Microsoft Office      \\ \hline
Programmation & C++, C, Python, Qt                                                                & Suivi de version & git                          \\ \hline
Calcul        & Matlab, Scilab                                                                    & Simulation       & Gazebo, Vortex, ADAMS \\ \hline
CAO           & \begin{tabular}[c]{@{}l@{}}Catia v5, Inventor, Solidworks,\\ FreeCAD\end{tabular} & CFD              & ANSYS Fluent                 \\ \hline
\end{tabular}
\nocite{*}
%
\bibliographystyle{abbrv}
\bibliography{publications}                       % 'publications' is the name of a BibTeX file
\section{Centres d'intérêts}
Vulgarisation scientifique, Jeux de société, Jeux de stratégie (échecs, jeux vidéo)
\end{document}