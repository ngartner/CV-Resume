\documentclass[11pt,a4paper]{moderncv}
\usepackage[french]{babel}\addto{\captionsfrench}{\renewcommand{\refname}{Bibliographie}}
\usepackage[utf8]{inputenc}
\usepackage{amsmath}
\usepackage{amsfonts}
\usepackage{amssymb}
\usepackage{xcolor}
\definecolor{Bvlt}{RGB}{78,51,255} 
%Marges et taille des colonnes de dates
\usepackage[left=2cm,right=2cm,top=1.1cm,bottom=1.1cm]{geometry}

\author{Nicolas Gartner}
\title{Curriculum Vitae}

% ma préférence en terme de design
\moderncvtheme[red]{classic}

% Une entête classique
\firstname{Nicolas}
\familyname{Gartner}
\title{Docteur-Ingénieur \newline Mécanique (fluide-solide) Robotique}
\extrainfo{28 ans \, Permis B}
\address{9 avenue Seyssaud}{13250 Saint Chamas}
\email{nicolas.gartner@gmail.com}
\mobile{+33 6 28 30 56 96}
\social[linkedin]{www.linkedin.com/in/ngartner}                        % optional, remove / comment the line if not wanted
\social[github]{ngartner}
\photo[70pt][0.4pt]{Nicolas_Gartner.jpg}

\begin{document}
\maketitle
% Marge négative entre le titre et la partie expérience, pour gagner de la place
\vspace*{-3\baselineskip}
\section{Expériences professionnelles}
\cventry{septembre-mars 2020}{Mission d'enseignement}{EM Lyon}{Ecully}{}{Enseignement de Python dans le cadre du master Digital Marketing \& Data Science.
\newline \textcolor{Bvlt}{Python, Pandas, Analyse de donnée, Anaconda,matplotlib}}
\cventry{aout-septembre 2020}{Consultant scientifique}{Centroid LAB}{Los Angeles (distanciel)}{USA}{Préparation de simulations de scénario catastrophe dans le cadre du projet open-source de centrale nucléaire Open-100
\newline \textcolor{Bvlt}{SPH, modélisation 3D, Python, CFD, hydrodynamique}}
\cventry{2019-mai 2020}{Ingénieur d'étude}{Laboratoire COSMER}{Université de Toulon}{}{Réalisation d'expérimentations dans le cadre du projet H2020 EU Marine Robots
\newline \textcolor{Bvlt}{ROS, programmation véhicule, Python, analyse des données, hydrodynamique}}
\cventry{2017-2019}{Chargé d'enseignement}{Seatech et IUT}{Université de Toulon}{}{Enseignement de la mécanique vibratoire TD/TP (104h) et de l'automatique TP (24h)
\newline \textcolor{Bvlt}{Pédagogie, Préparation des cours, Corrections, Mécanique des solides (déformable), contrôleur, PID}}
\cventry{2016}{Ingénieur R \& D}{Altran}{Aix-en-Provence}{}{Conception robotique : Projet Méthode et Analyse du Démantèlement Nucléaire (6 mois)
\newline \textcolor{Bvlt}{bras robotisé, conception, CAO, dimensionnement, Python, optimisation, Gazebo, ROS}}
\cventry{2015-2016}{Ingénieur R \& D}{Mayfran International}{Landgraaf}{Pays-Bas}{Développement produit / convoyeur et système de filtration (6 mois)
\newline \textcolor{Bvlt}{rétro-ingénierie, conception, CAO, écoulement, traduction technique}}
\cventry{2015}{Stage de recherche}{Universidade Federal Uberlândia}{}{Brésil}{Fabrication d'échangeur de chaleur par méthode de fabrication additive (6 mois) \newline
\textcolor{Bvlt}{Matlab, traitement de données, transfert thermique, Fluent, bras robotisé, soudage, métallurgie}}
\section{Formation}
\cventry{2016-2020}{Doctorat}{Université de Toulon}{Laboratoire COSMER}{}{Identification de paramètres hydrodynamiques par simulation avec Smoothed Particle Hydrodynamics.
\newline Membre élu au conseil du laboratoire COSMER, de l'école doctorale 548 et du pôle INP de l'université de Toulon
\newline \textcolor{Bvlt}{Simulation numérique, Écoulement (incompressible), Hydrodynamique, Dynamique des solides, Analyse numérique, Publications scientifiques, Paramètres hydrodynamiques, Interaction fluide-solide, robotique sous-marine}}
\cventry{2012-2016}{Diplôme d'ingénieur}{Sigma Clermont - ex IFMA}{Clermont-Ferrand}{}{Spécialisation Machines Mécanismes et systèmes}
%
\section{Langues}
\cvitemwithcomment{Français}{Langue maternelle}{}
\cvitemwithcomment{Anglais}{Courant}{TOEIC - 965 (2016)}
\cvitemwithcomment{Allemand}{Courant}{Goethe B2 - 91 (2016)}
\cvitemwithcomment{Portugais}{Avancé}{}
%
\clearpage
\section{Informatiques}
\renewcommand{\arraystretch}{1}
\setlength{\tabcolsep}{0.25cm}
\begin{tabular}{|l|l|l|l|}
\hline
OS            & Windows, Ubuntu, ROS        & Écriture         & LateX, Microsoft Office      \\ \hline
Programmation & C++, C, Python, Qt                                                                & Suivi de version & git                          \\ \hline
Calcul        & Matlab, Scilab                                                                    & Simulation       & Gazebo, Vortex, ADAMS \\ \hline
CAO           & \begin{tabular}[c]{@{}l@{}}Catia v5, Inventor, Solidworks,\\ FreeCAD\end{tabular} & CFD              & ANSYS Fluent                 \\ \hline
\end{tabular}
\nocite{*}
%
\section{Centres d'intérêts}
Vulgarisation scientifique, Jeux de société, Jeux de stratégie (échecs, jeux vidéo)

\bibliographystyle{abbrv}
\bibliography{publications}                       % 'publications' is the name of a BibTeX file

\end{document}